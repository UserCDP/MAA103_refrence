\section{Modular arithmetic}
        \subsection{Modular addition and multiplication}
            Let $n\in\mathbb{N}, n \geq 2$. $\mathbb{Z}/n\mathbb{Z} = \{ [0],...,[n-1]\}$, where $[x]$ denotes equivalence calss of $x\,mod(n)$.

            \subsubsection{Addition}
                We define on $\mathbb{Z}/n\mathbb{Z}$, the addition law $\oplus$ by $[x] \oplus [y] = [x+y] = x + y$ $mod(n)$. $(\mathbb{Z}/n\mathbb{Z}, \oplus)$ is an Abelian group; [0] is the neutral element, and the inverse of [x] is [-x].

                \vspace{5pt}
    
                $\bigstar$ the operation $\oplus$ depends on \textbf{n}
    
            \subsubsection{Prop}
                Let $a,b \in \mathbb{Z}$ and $n \geq 2$ integer. If $x \equiv a\,mod(n)$ and $y \equiv b\,mod(n)$, then $xy \equiv ab\,mod(n)$.
    
            \subsubsection{Multiplication}
                We define \textbf{multiplication} on $\mathbb{Z}/n\mathbb{Z}$ by $[a][b]=[ab]$.

        \subsection{The Chinese remainder theorem}
            Let $a,b,m,n \in \mathbb{Z}$, with $m \geq 2, n \geq 2$, and $m,n$ coprime. Then there is a unique integer $x_0$, $0\leq x_0<mn$ that solves the system of equations:

            \begin{equation}
                \begin{cases}
                    x \equiv a$ $mod(m) \\
                    x \equiv b$ $mod(n)
                \end{cases}
            \end{equation}

            \vspace{5pt}
            
            $\bigstar$ Moreover, any solution of $(1)$ satisfies $z \equiv x_0$ $mod(mn)$

            \newpage

            \subsubsection{Generalization}
                Let $n_1,...,n_k \in \mathbb{N}$ with $n_i\geq 2$, and pairwise coprime; let $a_1,...,a_k \in \mathbb{Z}$. Then $\exists ! a \in \mathbb{Z}$ with $0\leq a < n_1,...,n_k$

                    \begin{align*}
                        \begin{cases}
                            x \equiv a_1 \, mod(n_1) \\
                            \vdots \\
                            x \equiv a_k \, mod(n_k)
                        \end{cases}
                        & \Leftrightarrow x \equiv a \, mod(n_1...n_k)
                    \end{align*}

        \subsection{Fermat's little theorem}
            \subsubsection{Theorem}
                Let $p$ be a prime number, and let $a \in \mathbb{Z}$ such that $p$  does not diveide $a$. Then 

                \vspace{5pt}
                
                \centerline{$a^{p-1} \equiv 1 \, mod(p)$}

            \subsubsection{Remark}
                Equivalent formula: $\forall a \in \mathbb{Z}, a^p \equiv a \, mod(p)$; indeed, if $p|a$ then $a\equiv 0 \, mod(p)$ so $a^p \equiv 0 \equiv a \, mod(p)$, if $p\nmid a$ then $a$ and $p$ are coprime so $[a]^{-1}$ exists in $(\mathbb{Z}/p\mathbb{Z})^\ast$ and $[a]^p = [a]$ is equivalent to $[a]^{p-1} = 1$.

                \vspace{5pt}

                $\bigstar$ there exist non prime numbers satisfying the conclusion of the theorem (it is not an equivalence)

            \subsubsection{Remark}
                The theorem generalizes (Euler theorem) to the case $n \geq 2$ is an integer and $a$ and $n$ are coprime. In this case, we have $a^{\phi(n)} \equiv 1 \, mod(n)$, where $\phi(n)$ is the number of integers between 1 and $n$ that are coprime with $n$, i.e. the number of invertible elements in $(\mathbb{Z}/n\mathbb{Z}, x)$.

            \subsubsection{Application of the theorem}
                It gives a (partial) criterion to prove that a given number (possibly large) is not prime: indeed, if we can find $a \in \mathbb{Z}$ such that $a^n \not\equiv a \, mod(n)$ then n is not prime. Note that computing $a^n \, mod(n)$ may be much faster than decomposing $n$ into prime factors.